\documentclass[letterpaper, 12pt]{article}
\newcommand{\myname}{Oliver Kis}
\newcommand{\mystudentnumber}{15576333}
\newcommand{\hw}{3} %%This needs to be changed each assignment
\newcommand{\dates}{02 February 2024}


%Math Packages 
\usepackage{amsmath, amsfonts, amssymb, amsthm}
\usepackage{setspace}
\usepackage[usenames, dvipsnames]{color}
\usepackage{nicefrac}
\theoremstyle{definition}
\newtheorem{pb}{Problem} %%shortcut for problem word


%photo packages
\usepackage{graphicx} % Required for inserting images


\usepackage[paper = letterpaper, left = 25mm, right = 25mm, top = 3cm, bottom = 25mm]{geometry}

\usepackage{fancyhdr}
\pagestyle{fancy}


\lhead{\myname}
\chead{Homework \hw}
\rhead{\mystudentnumber}

\lfoot{MATH 220}
\cfoot{Page \thepage}
\rfoot{\mystudentnumber}

\renewcommand{\headrulewidth}{0.4pt}
\renewcommand{\footrulewidth}{0.4pt}



\begin{document}
\begin{titlepage}
\centering
    \includegraphics{UBC.png}\\
    \vspace{1cm}
    {\bfseries\large
    Homework \hw \\
    \vspace{0.5cm}
    \myname\\
    \vspace{0.5cm}
    \mystudentnumber\\
    \vspace{0.5cm}
    \dates\\  %%THIS DATE NEEDS TO BE CHANGED ALWAYS
    }
    
    \vfill
    {\large
    The University of British Columbia\\
    \vspace{0.25cm}
    Electrical and Computer Engineering\\
    \vspace{0.25cm}
    Math 220 - Mathematical Proofs\\
    \vspace{0.25cm}
    Instructor: Bennett Michael\\
    }


\end{titlepage}

\subsection*{Solutions}
For Homework \hw, Problem 3, 4, 5, 6, a part of 7, and a part of 8 are worth marks.
\begin{spacing}{1.33}


%%Problem 3
\vspace{5mm}
\setcounter{pb}{2}
\begin{pb} 


    $\color{blue}{(2}$ $\color{blue}{Points)}$ Let $a \in \mathbb{Z}$. Prove the following statement:
    \begin{center}
        if $5 \mid 2a$, then $5 \mid a$.
    \end{center}  
    


    %Answer to Problem 3
    Proof: Let $a \in \mathbb{Z}$. Using divisibility rules and implication rules, if the hypothesis and conclusion are both true, then the implication is true.
    \begin{center}
        if $2a = 5k$ where $k \in \mathbb{Z}$\\
        $a = \frac{5k}{2}$\\
        Where $a \in \mathbb{Z} $, this means $k = 2m$\\
        $a = \frac{5 * 2m}{2}$= $5m$
        
    \end{center}



\end{pb}




%%Problem 4
\vspace{20cm}
\begin{pb} 
    $\color{blue}{(5}$ $\color{blue}{Points)}$
    \begin{enumerate}

        %%Problem 4-1
        \item[1.] $\color{blue}{(1}$ $\color{blue}{Points)}$ Prove the following statement. For every   $a \in \mathbb{R}$,
        \begin{center}
            if $a \geq 4$, then $-\nicefrac{a^2}{4} + a \leq 0$.
        \end{center}
    
        %Answer for Problem 4-1
        Proof: Let $a \in \mathbb{R}$. Solve via contrapositive.
        \begin{center}
            if $-\frac{a^2}{4} + a > 0$ then $a < 4$\\
            $-\frac{a^2}{4} > -a \longrightarrow \frac{a^2}{4} < a$\\
            $a^2 < 4a = a < 4$
        \end{center}
    
    
    
        %%Problem 4-2
        \item[2.] $\color{blue}{(4}$ $\color{blue}{Points)}$ Let $a \in \mathbb{R}$. Prove the  following statement:
        \begin{center}
            (for every $x \in \mathbb{R}$, we have $x^2 +ax+a > 0$) if and only if $(0 < a < 4)$.
        \end{center}
    
        %%Proof for PRoblem 4-2
        Proof: Let $a \in \mathbb{R}$. Biconditionally solve this problem.
        \begin{center}
            $\forall$  $x \in \mathbb{R}$, $x^2 + ax + a > 0 \Longleftrightarrow 0 < a < 4$\\
            \vspace{5mm}
            $(\Longleftarrow)$ : Completeting the square first $\rightarrow$ $(x + \frac{a}{2})^2 + \frac{4a-a^2}{4} > 0$\\
            Since $0 < a < 4 \therefore 4a - a^2 > 0$, thus $\frac{4a-a^2}{4} > 0$\\
            $(x + \frac{a}{2})^2 \geq 0$ since $()^2$ is always positive\\
            Hence, $(x + \frac{a}{2})^2 + \frac{4a-a^2}{4} > 0$ when $0 < a < 4$\\
            \vspace{5mm}
            $(\Longrightarrow)$ : Utilizing the contrapositive from part one to solve this.\\
            $4a - a^2 = -\frac{a^2}{4} + a \longrightarrow -\frac{a^2}{4} + a > 0$ from earlier proof.\\
            $\therefore 4a - a^2 > 0$ and reworking $a < 4$.\\
            Thus, our range for $a$ is $0 < a < 4$ for $\frac{4a - a^2}{4}$\\
            Additionally, $(x + \frac{a}{2})^2 \geq 0$ since $()^2$ is always positive, but it can be zero\\
            Finally, if $(x + \frac{a}{2})^2$ is zero $\longrightarrow 0 + \frac{4a-a^2}{4} >0$, this implies to get $>0$, we need $0<a<4$\\

            \vspace{5mm}
            With both implications being solved, the biconditional statement is true.
        \end{center}
        
    \end{enumerate}
\end{pb}





%%Problem 5
\vspace{1cm}
\begin{pb} 
    $\color{blue}{(2}$ $\color{blue}{Points)}$ Let $m$ $\in$ $\mathbb{Z}$. Prove that if $5 \nmid m$, then $m^2 \equiv 1$ (mod $5)$ or \\ 
    $m^2 \equiv -1$ (mod $5)$.

    %%Answer to Problem 5
    Proof: This is solved by proof by cases. Let $m \in \mathbb{Z}$. When $5\nmid m$, there are 4 cases.

    \begin{center}
        1) Case 1 : $m\equiv1(mod5) \longrightarrow m =5k+1$ for some $k \in \mathbb{Z}$\\
        $m^2 = (5k+1)^2 \longrightarrow m^2 = 25k^2 + 10k +1 \longrightarrow m^2 = 5(5k^2+2k)+1$\\
        $\therefore m^2 \equiv 1(mod5)$, which holds our first statement true\\
        \vspace{5mm}
        2) Case 2 : $m\equiv2(mod5) \longrightarrow m^2 = (5k+2)^2 \longrightarrow m^2 = 5(5k^2 + 4k) + 4$\\
        $\therefore m^2 \equiv 4(mod5)$ which is equivalent to $m^2 \equiv -1(mod5)$\\
        \vspace{5mm}
        3) Case 3 : $m\equiv3(mod5)$ achieves the same as Case 2\\ $m^2 \equiv 9(mod5) = m^2 \equiv -1(mod5)$\\
        \vspace{5mm}
        4) Case 4 : $m\equiv4(mod5)$ achieved the same as Case 1\\ $m^2 \equiv 16(mod5) = m^2 \equiv 1(mod5)$\\
        \vspace{5mm}
        Thus, we proved that $5 \nmid m$, then $m^2 \equiv 1$ (mod $5)$ or $m^2 \equiv -1$. (mod $5)$
    \end{center}

\end{pb}






%% Problem 6
\vspace{1cm} %%Skip to next page
\begin{pb}
    $\color{blue}{(2}$ $\color{blue}{Points)}$ For $a \in \mathbb{Z}$, prove:
    \begin{center}
    $3 \nmid a \Longrightarrow$ (there exists $b \in \mathbb{Z}$ such that $ab \equiv 1 mod 3$
    \end{center}

    %Proof to Problem 6
    Proof: Let $a, b \in \mathbb{Z}$. This is solved by two cases
    \begin{center}
            1) Case 1 : $a = 3k + 1 \Longrightarrow \exists \: b \in \mathbb{Z} \: s.t. \:  ab = 3k + 1$\\
            Using simple numbers, when $b = 1 \longrightarrow a = 3k + 1 \Longrightarrow a\equiv 1$(mod3).\\
            \vspace{5mm}
            2) Case 2 : $a = 3k + 2 \Longrightarrow \exists \: b \in \mathbb{Z} \: s.t. \:  ab = 3k + 1$\\
            When $b = 2$, $a = 3k +2 \longrightarrow 2a = 2(3k + 2) = 6k + 4 = 6k + 3 + 1 = 3(2k + 1) + 1$\\
            $\therefore$ when $b = 2, ab \equiv 1$(mod3)
            
    \end{center}
\end{pb}


%% Problem 7-1
\vspace{5cm}
\begin{pb}
    $\color{blue}{(2}$ $\color{blue}{Points)}$ Prove that the product of 5 consecutive integers is a multiple of 5.\\

    %% Answer to 7-1

    Proof: Pure logic solves this question. Since we utilize 5 consecutive integers, this indicates at one point there is a multiple of 5 in the product. This means the product will be divisible by 5.

    \begin{center}
        $a, b, c, d, e \in \mathbb{Z}$. We know with consecutive numbers that $a=a, b = a+1, c = a+2, d = a+3,$ and $e = a + 4$.\\
        Testing a value such as $a=1$ or $a=2$, this stands true.\\
        Since we know our hypothesis is true, it also stands that $a=a+1, b = a+2, c = a+3, d = a+4,$ and $e = a + 5$\\
        $e = a + 5$ has a remainder of $5$, thus as long as one part of the product is divisble by 5, all of it is.
        
    \end{center}

    
\end{pb}


%% Problem 8-1

\begin{pb}
    $\color{blue}{(2}$ $\color{blue}{Points)}$ We recall that given $a, b \in \mathbb{Z}$ such that $ab \neq 0$, we define the gcd of $a$ and $b$ to be the greatest integer that divides both $a$ and $b$. We denote this by $gcd(a,b)$\\
    
    Let $a, b \in \mathbb{Z}$ such that $ab \neq 0$. We suppose that there exists $u, v, \in \mathbb{Z}$ such that
    \begin{center}
        $1 = au + bv$
    \end{center}
    Prove that gcd$(a, b) \equiv 1$.\\


    %%Answer to 8-1
    Proof: Let $a, b \in \mathbb{Z}$. We also know neither $a$ or $b$ can be $0$. Researching B\'ezout's identity and greatest common divisor, we know that $c \in \mathbb{Z}$ divides $gcd(a,b)$. We also know that $c$ also divides $au$ and $bv$. Through linearity, we know when added, $au + bv = 1$. This also states that 1 is divisible by $c$. If you divide 1, you can only divide by 1 or -1. But gcd is always going to assume a positive value.
    $\therefore gcd(a,b) \equiv 1$. Additionally, that means $a, b$ are primes.
\end{pb}


\end{spacing}
\end{document}


