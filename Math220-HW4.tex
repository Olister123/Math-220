\documentclass[letterpaper, 12pt]{article}
\newcommand{\myname}{Oliver Kis}
\newcommand{\mystudentnumber}{15576333}
\newcommand{\hw}{4} %%This needs to be changed each assignment
\newcommand{\dates}{09 February 2024}


%Math Packages 
\usepackage{amsmath, amsfonts, amssymb, amsthm}
\usepackage{setspace}
\usepackage[usenames, dvipsnames]{color}
\usepackage{nicefrac}
\theoremstyle{definition}
\newtheorem{pb}{Problem} %%shortcut for problem word


%photo packages
\usepackage{graphicx} % Required for inserting images


\usepackage[paper = letterpaper, left = 25mm, right = 25mm, top = 3cm, bottom = 25mm]{geometry}

\usepackage{fancyhdr}
\pagestyle{fancy}


\lhead{\myname}
\chead{Homework \hw}
\rhead{\mystudentnumber}

\lfoot{MATH 220}
\cfoot{Page \thepage}
\rfoot{\mystudentnumber}

\renewcommand{\headrulewidth}{0.4pt}
\renewcommand{\footrulewidth}{0.4pt}



\begin{document}
\begin{titlepage}
\centering
    \includegraphics{UBC.png}\\
    \vspace{1cm}
    {\bfseries\large
    Homework \hw \\
    \vspace{0.5cm}
    \myname\\
    \vspace{0.5cm}
    \mystudentnumber\\
    \vspace{0.5cm}
    \dates\\  %%THIS DATE NEEDS TO BE CHANGED ALWAYS
    }
    
    \vfill
    {\large
    The University of British Columbia\\
    \vspace{0.25cm}
    Electrical and Computer Engineering\\
    \vspace{0.25cm}
    Math 220 - Mathematical Proofs\\
    \vspace{0.25cm}
    Instructor: Bennett Michael\\
    }


\end{titlepage}

\subsection*{Solutions}
For Homework \hw, Problem 1, 2, parts of 4, and parts of 5 are worth marks.
\begin{spacing}{1.33}


%%Problem 1
\vspace{5mm}
\setcounter{pb}{0}
\begin{pb} 


    $\color{blue}{(2}$ $\color{blue}{Points)}$ Prove, for $x, y, \in \mathbb{Z}$, that
    \begin{center}
        $(xy$ even and $x + y$ even) $\Longrightarrow (x$ even and $y$ even)
    \end{center}  
    


    %Answer to Problem 1
    Proof: Let $x, y \in \mathbb{Z}$. We proved this by Contrapositive and several cases. The \newline \indent contrapositive turns out to be $(x$ odd or $y$ odd) $\Longrightarrow (xy$ odd or $x+y$ odd)
    \begin{itemize}
        \item[1)] Case one 

            {\centering $x$ is odd and $y$ is odd $\therefore$ $x = 2k + 1$ and $y = 2l + 1$ where $k, l \in \mathbb{Z}$.
            $xy = (2k+1)(2l+1) = 4kl + 2k + 2l + 1 = 2(2kl + k + l )+1$\par}
            
        \item[2)] Case two

            {\centering $x$ is odd and $y$ is even $\therefore$ $x = 2k + 1$ and $y = 2l$ where $k, l \in \mathbb{Z}$. 
            $x + y = 2k + 1 + 2l = 2 (k + l) + 1$ \par}
            
        \item[3)] Case three
        
            {\centering Identical to case two, but $x$ and $y$ are "flipped" $x$ is even and $y$ is odd.\par}
    \end{itemize}

    Hence, all three cases prove the contrapositive, which proves the original statement, as required. //

\end{pb}


%%problem 2
\vspace{5mm}
\begin{pb}
    $\color{blue}{(2}$ $\color{blue}{Points)}$ For $a \in \mathbb{R}$, we define the set $S_a = {x \in \mathbb{R} : (x \geq 0 \wedge x < a - 2)}$.\\
    Show that
    
        {\centering $S_a = \emptyset$ if and only is $a \in ] - \infty, 2]$ \par}
    
    %%Solution
    Proof: Let $a, x \in \mathbb{R}$. To prove this, we use the biconditional technique.\\

    {\centering $\Longrightarrow$ If our set is null, we know $x \geq 0$ isn't satisfied and/or $x < a - 2$ isn't satisfied. \\ Using the bounds, when $a = 2$, we know $x < 0$, which doesn't hold with the statement $x \geq 0$ \par}
    {\centering $\Longleftarrow$ With $a \in ] - \infty, 2]$, we know $-\infty < a-2 \leq 0$. Using the maximum, this won't satisfy both $x \geq 0$ and $x < a-2$  \par}

    
\end{pb}






%%problem 3
\vspace{5mm}
\setcounter{pb}{3}
\begin{pb}
    $\color{blue}{(12}$ $\color{blue}{Points)}$ For each of the following statements:
    \begin{itemize}
        \item Negate the statement.
        \item Decide if the original statement is true or false and justify your answer.
    \end{itemize}
    \begin{enumerate}
        \item[1.] $\color{blue}{(2}$ $\color{blue}{Points)}$  $\forall \: a \in \mathbb{Z}, ((6\mid a$ and $8\mid a \: \Longrightarrow 48 \mid a).$
        \begin{itemize}
            \item $\exists \: a \in \mathbb{Z}$, $((6 | a$ and $8 | a \Longrightarrow 48 \nmid a)$.
            \item False $\longrightarrow$ Finding gcd(6, 8) is 24, next multiple is 48. These work with the implications, however the next multiple of 72 does not.  
        \end{itemize}
        
        \item[2.] $\color{blue}{(2}$ $\color{blue}{Points)}$ $ \forall \: x \in \mathbb{R}, \forall y \in \mathbb{R}, (xy \geq 0 \Longrightarrow x + y \geq 0).$
        \begin{itemize}
            \item $\exists \: x \in \mathbb{R}, \exists \: y \in \mathbb{R} $ s.t. $(xy \geq 0 \Longrightarrow x + y < 0).$
            \item False $\longrightarrow$ $\forall \: (-x)$ and $\forall \: (-y)$ satisfy $xy \geq 0$, but does not satisfy $(x + y \geq 0).$ 
        \end{itemize}
        
        \item[3.] $\color{blue}{(2}$ $\color{blue}{Points)}$ $\forall \: a, b \in \mathbb{Z}, \forall n \in \mathbb{N}, (6a \equiv 6b$ mod $6n \Longrightarrow a \equiv b$ mod $n$).
        \begin{itemize}
            \item $\exists \: a, b \in \mathbb{Z}, \exists \: n \in \mathbb{N}, (6a \equiv 6b$ mod $6n \Longrightarrow a \not\equiv b$ mod $n$).
            \item  False $\longrightarrow 6a \equiv 6b$ mod $6n$ is equivalent to $6a = 6nk + 6b$, where $k \in \mathbb{Z}$, divide both sides by 6, $a = nk + b$ which is equivalent to $a = b$ mod $n$. But, with $\forall \: a, b$, when $a$ is divided by $nk$, there is a time when $b$ is just zero, not all values.
        \end{itemize}

        \item[4.] $\color{blue}{(2}$ $\color{blue}{Points)}$ $\forall \: a, b \in \mathbb{Z},  (4a \equiv 4b$ mod $24 \Longrightarrow a \equiv b$ mod $24$). 
        \begin{itemize}
            \item $\exists \: a, b \in \mathbb{Z}, (4a \equiv 4b$ mod $24 \Longrightarrow a \not\equiv b$ mod $24)$.
            \item False $\longrightarrow$ As shown before this, when $4a$ is divided to have a remainder of zero, there should be no other value of $b$ than zero.
        \end{itemize}

        \item[5.] $\color{blue}{(2}$ $\color{blue}{Points)}$ $\exists \: x \in \mathbb{Z}$ such that $((x > 84) and (x \equiv 75$ mod $84$)).
        \begin{itemize}
            \item $\forall \: x \in \mathbb{Z}$ such that $((x \leq 84$ or $x \not\equiv 75$ mod84)).
            \item True $\longrightarrow$ When x is above 84, specifically when x is 159, the statement holds. 
        \end{itemize}
        
        \item[6.] $\color{blue}{(2}$ $\color{blue}{Points)}$ $\exists \: x, y \in \mathbb{R}$ such that $(x^2 \geq y^2$ and $x \leq y)$.
        \begin{itemize}
            \item $\forall \: x, y \in \mathbb{R}$ such that $(x^2 < y^2$ or $x > y)$.
            \item True $\longrightarrow$ Take [-1, 1] and the original statement works
        \end{itemize}
    \end{enumerate}
\end{pb}




%%Problem 5
\vspace{10cm}
\begin{pb}
    $\color{blue}{(4}$ $\color{blue}{Points)}$ Let $(u_0, u_1, u_2, u_3, ... )$ be a sequence of real numbers. We write this as $(u_n)_{n \in \mathbb{N}}.$ We say that it is:
    \begin{itemize}
        \item bounded above when: $\: \exists \: A \in \mathbb{R} \: s.t. ( \forall \: n \in \mathbb{N}, u_n \leq A).$
        \item bounded below wheb: $\: \exists \: B \in \mathbb{R} \: s.t. ( \forall \: n \in \mathbb{N}, u_n \geq B).$
    \end{itemize}
    We say that it converges towards $+\infty$ when
    \begin{center}
        $\forall \: A > 0, \exists \: m \in \mathbb{N} \: s.t. \forall \: n \in \mathbb{N}, (n \geq m \Longrightarrow u_n > A).$
    \end{center}
    and that it converges towards $-\infty$ when
    \begin{center}
        $\forall \: B < 0, \exists \: m \in \mathbb{N} \: s.t. \forall \: n \in \mathbb{N}, (n \geq m \Longrightarrow u_n < B).$
    \end{center}
    \begin{enumerate}
        \item [1.]$\color{blue}{(1}$ $\color{blue}{Points)}$ Write in quantifiers the statement:
        \begin{center}
            $(u_n)_{n \in \mathbb{N}}$ is not bounded below.
        \end{center}
        \item [2.]$\color{blue}{(1}$ $\color{blue}{Points)}$ Give an example of sequence of real numbers $(u_n)_{n \in \mathbb{N}}$ which is bounded above but not bounded below.
        \item [3.]$\color{blue}{(2}$ $\color{blue}{Points)}$ Write in quantifiers the statement:
        \begin{center}
            $(u_n)_{n \in \mathbb{N}}$ does not converge towards $+\infty$.
        \end{center}
    \end{enumerate}
\end{pb}



\end{spacing}
\end{document}